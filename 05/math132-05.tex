\documentclass{../homework}

\homework{5}
\due{noon on Friday 2/29 in the Math 132 mailbox}
\author{}
\coauthor{}

\begin{document}
\begin{Exercise}
	Show that $S = \vecspan \{ f(x)g(y) : f,g\in C[0,1]\}$ is dense in
  $C( [0,1]^2)$.\footnote{$[0,1]^2=[0,1] \times [0,1] \subset \R^2$.
    This problem is related to \emph{tensor products} ($\otimes$), a
    general mechanism for taking the product of two mathematical
    structures.  For instance, $\R^n \otimes \R^n = \mathsf{M}_n(\R)$,
    the set of $n \times n$ matrices (the equality is only up to
    isomorphism, but we can ignore that now).  In fancier terminology,
    you must show that $C[0,1] \otimes C[0,1] = C( [0,1]^2)$ (up to
    some notion of isomorphism).}

  \begin{solution}

  \end{solution}
\end{Exercise}

\begin{Exercise}
	In this exercise, we prove that if $X$ is compact, then $C(X)$ is
  separable.\footnote{A proof based upon the Arzel\`a--Ascoli Theorem
    was presented in lecture.}
	\begin{enumerate}
  \item Let $X$ be a compact metric space and let
    $S = \{x_1,x_2,\ldots\}$ be a countable dense subset of
    $X$.\footnote{Compact metric spaces are separable (lecture).}  For
    $n \in \N$, let $f_n(x) = d(x,x_n)$ and let $\A$ denote the
    algebra generated by $\{1, f_1,f_2,f_3,\ldots\}$.\footnote{In
      other words, $\A$ is the collection of all finite products of
      linear combinations of $1, f_1,f_2,f_3,\ldots$.}  Prove that
    $\A$ is dense in $C(X)$.

    \begin{solution}

    \end{solution}

  \item Prove that if $X$ is compact, then $C(X)$ is separable.
    \footnote{Hint: Consider the set $\A'$ of all finite products of
      \emph{rational} (finite) linear combinations of
      $1, f_1,f_2,f_3,\ldots$.  Remember that a compact metric space
      is bounded.}

    \begin{solution}

    \end{solution}
	\end{enumerate}
\end{Exercise}

\begin{Exercise}
	Let $X$ be a compact metric space.  Then $S \subseteq C(X)$
  \emph{vanishes at $a \in X$} if $f(a) = 0$ for all $f \in S$.  A
  subalgebra $I \subseteq C(X)$ is an \emph{ideal} in $C(X)$ if
	\begin{equation*}
		f\in C(X), \,\,g\in I \quad \implies \quad fg \in I.
	\end{equation*}
	An ideal $I$ is \emph{proper} if $I \neq C(X)$.  A proper ideal $I$
  is \emph{maximal} if whenever $J$ is an ideal of $C(X)$ and
  $I \subseteq J$, either $J = I$ or $J = C(X)$.  \footnote{Note that
    a maximal ideal must be proper!}
  \begin{enumerate}
  \item Suppose that $\A$ is a subalgebra of $C(X)$ that does not
    vanish at any point of $X$.  Prove that there is a $g \in \A$ so
    that $g(x) > 0$ for all $x \in X$.  \footnote{\emph{Hint}: Prove
      that there are $f_1,f_2,\ldots, f_n \in \A$ such that
      $g(x) = \sum_{i=1}^n |f_i(x)|^2 > 0$ for all $x \in X$.}

    \begin{solution}

    \end{solution}

  \item Use (i) to prove that every maximal ideal of $C(X)$ is of the
    form
    \begin{equation*}
      I_a = \{ f \in C(X) : f(a) = 0 \}
    \end{equation*}
    for some $a \in X$.\footnote{Thus, there is a correspondence
      between $X$ and the set of maximal ideals of $C(X)$; the
      algebraic structure of $C(X)$ encodes the topological structure
      of $X$.  These are the first steps towards Israel Gelfand's
      far-reaching theory of commutative Banach algebras.}

    \begin{solution}

    \end{solution}
  \end{enumerate}
\end{Exercise}

\begin{Exercise}
	Let $f_n \in C^1[0,1]$ and suppose that
	\begin{align*}
		|f_n'(x)| &\leq \frac{1}{\sqrt{x}}  &\text{for $x \in (0,1]$},\\
		\int_0^1 f_n(x)\,dx &=0 & \text{for all $n \in \N$.}
	\end{align*}
	Prove that the sequence $f_n$ has a subsequence that converges
  uniformly on $[0,1]$.\footnote{Hint: Use the Arzel\`a--Ascoli
    Theorem.  For equicontinuity, use the Fundamental Theorem of
    Calculus and then use properties of the function
    $F(x) = 2\sqrt{x}$.  For boundedness, first try showing that
    $\int_0^1 f_n(x)\,dx = 0$ implies that there exists
    $x_n \in [0,1]$ such that $f_n(x_n) = 0$. }

  \begin{solution}

  \end{solution}
\end{Exercise}

\begin{Exercise}
	Let $\T =\{ z \in \C : |z| =1 \}$ denote the unit
  circle\footnote{The use of the symbol $\T$ to denote the unit circle
    may appear strange.  The letter ``T'' stands for ``torus'' -- a
    doughnut shaped two-dimensional surface.  The unit circle is
    viewed as a ``one-dimensional torus.''  This is not as strange as
    it sounds -- topologically speaking the Cartesian product
    $\T \times \T$ is homeomorphic to a torus (try to understand
    why!).  In general, $\T^n$ is regarded as an $n$-dimensional
    torus.} in the complex plane $\C$.  Let $C(\T)$ denote the set of
  all continuous complex-valued functions $f: \T \to \C$.  Define an
  inner product on $C(\T)$ by
	\begin{equation*}
		\inner{f,g}
    = \frac{1}{2\pi} \int_{-\pi}^{\pi}
    f(e^{it}) \overline{ g(e^{it}) }\,dt.
	\end{equation*}
	The reason that the $g$ is conjugated\footnote{Recall that if
    $z = a+bi$ ($a,b \in \R$), then $\overline{z} = a-bi$.  Moreover,
    also note that $z \overline{z} = |z|^2$.}  is to ensure that
	\begin{equation*}
		\norm{f}_2^2
    = \inner{f,f}
    = \frac{1}{2\pi} \int_{-\pi}^{\pi} f(e^{it}) \overline{f(e^{it})}\,dt
    = \frac{1}{2\pi} \int_{-\pi}^{\pi} |f(e^{it})|^2\,dt \geq 0.
	\end{equation*}
	Other than this minor difference, inner products on complex vector
  spaces behave much the same as their real counterparts.
	\begin{enumerate}
  \item Show that the functions
    $\ldots, \overline{z}^2, \overline{z}, 1 , z , z^2, \ldots$ are
    orthonormal in $C(\T)$.\footnote{Hint: You may assume that complex
      integration works much the same as real integration.  }

    \begin{solution}

    \end{solution}

  \item Let $\C[z]$ denote the set of all polynomials
    $p(z) = \sum_{j=0}^n a_j z^j$ in the variable $z$ with complex
    coefficients.  Prove that $\inner{p(z), \overline{z}} = 0$ for all
    $p(z) \in \C[z]$.

    \begin{solution}

    \end{solution}

  \item Prove that $\norm{g}_2 \leq \norm{g}_{\infty}$ for all
    $g \in C(\T)$.

    \begin{solution}

    \end{solution}

  \item Prove that $\norm{\overline{z} - p(z)}_{\infty} \geq 1$ for
    all $p(z) \in \C[z]$.  Although $\C[z]$ is an algebra of
    continuous functions on $\T$ that contains $1$ (the constant
    function) and separates points, $\C[z]$ is not dense in
    $C(\T)$.\footnote{Nonconstant polynomials deform the complex plane
      in a way that \emph{preserves orientation} -- things get
      distorted, but preserve their ``handedness.''  On the other
      hand, the function $\overline{z}$ is a ``flip'' with respect to
      the $x$-axis (i.e., the real axis) and hence \emph{reverses
        orientation}.  This is why $\overline{z}$ cannot be
      approximated too closely by polynomials in $z$.  The complex
      Stone--Weierstrass Theorem adds a fourth hypothesis:
      \textbf{``(iv) $\A$ must be closed under complex
        conjugation.''}}

    \begin{solution}

    \end{solution}
	\end{enumerate}
\end{Exercise}
\end{document}