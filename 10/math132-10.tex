\documentclass{../homework}

\homework{10}
\due{in the Math 132 mailbox before 2pm on Friday 4/12}

\author{}
\coauthor{}

\begin{document}
\begin{Exercise}
  This exercise outlines a proof of a fundamental result in the theory
  of Fourier series.  Specifically, we wish to prove that if
  \(f \colon \T \to \C\) is Riemann integrable and
  \(\widehat{f}(n) = 0\) for all \(n \in \Z\), then \(f(x) = 0\)
  whenever \(f\) is continuous at \(x\).
  \begin{enumerate}
  \item Suppose toward a contradiction that \(\hat f (n) = 0\) for all
    \(n \in \Z\), yet \(f(x_0) \neq 0\) at some point \(x_0\) at which
    \(f\) is continuous.  Prove that it suffices to consider the case
    \(x_0 = 0\).\footnote{\emph{Hint}: Consider the function
      \(g(x) = f(x+x_0)\).}

    \begin{solution}
      \begin{proof}

      \end{proof}
    \end{solution}

  \item Prove that it suffices to consider the case where
    \(f \colon \T \to \R\) and \(f(0) > 0\).\footnote{\emph{Hint}:
      First prove that
      \(\hat f (n) = 0 \;\; \forall n \in \Z \iff \hat{\bar f}(n) = 0
      \;\; \forall n \in \Z\).}

    \begin{solution}
      \begin{proof}

      \end{proof}
    \end{solution}

  \item Under the assumptions above, prove that
    \[
      \frac{1}{2\pi} \int_{-\pi}^{\pi} f(x) q(x) \dif x = 0
    \]
    for any trigonometric polynomial
    \(q(x) = \sum_{n = -N}^M a_n e^{inx}\).

    \begin{solution}
      \begin{proof}

      \end{proof}
    \end{solution}

  \item Since \(f\) is continuous at \(0\) and \(f(0) > 0\), there
    exists \(0 < \delta \leq \frac{\pi}{2}\) such that
    \(f(x) > \frac{f(0)}{2}\) whenever \(\abs x < \delta\).  Define
    the trigonometric polynomial \(p(x) = \epsilon + \cos x\), in
    which \(\epsilon > 0\) is so small that
    \begin{equation*}
      \abs{p(x)} < 1 - \frac{\epsilon}{2}
    \end{equation*}
    whenever \(\delta \leq \abs x \leq \pi\).  Choose a positive
    \(\eta < \delta\) so that
    \begin{equation*}
      1 + \frac{ \epsilon}{2} \leq \abs{p(x)}
    \end{equation*}
    for \(\abs x < \eta\).  Finally, let \(p_n(x) = [p(x)]^n\) for
    \(n \in \N\).  Carefully illustrate the construction of \(p(x)\).

    \begin{solution}

    \end{solution}

  \item Since each \(p_n\) is a trigonometric polynomial, it follows
    from (iii) that\footnote{The technique used in (iv) and (v) is a
      standard one -- construct a sequence of functions which ``peak''
      at just the right place and which ``die off'' elsewhere.}
    \begin{equation*}
      \int_{-\pi}^{\pi} f(x) p_n(x)\dif x = 0
    \end{equation*}
    for \(n \in \N\).  Estimate the integrals \(I_1,I_2,I_3\) in the
    decomposition
    \begin{equation}
      \label{eq-TripleSum}
      \underbrace{\int_{0 \le \abs x < \eta}
        f(x) p_n(x) \dif x}_{I_1}
      + \underbrace{\int_{\eta \le \abs x < \delta}
        f(x) p_n(x) \dif x}_{I_2}
      + \underbrace{\int_{\delta \le \abs x \le \pi}
        f(x) p_n(x) \dif x}_{I_3} = 0
    \end{equation}
    and arrive at a contradiction.

    \begin{solution}
      \begin{proof}

      \end{proof}
    \end{solution}
  \end{enumerate}
\end{Exercise}

\begin{Exercise}
  For \(N \in \N\), the \emph{Dirichlet kernel} is the trigonometric
  polynomial
  \(D_N(x) = \sum_{n=-N}^N e^{inx} = 1 + 2 \sum_{n=1}^N \cos nx\).
  \begin{enumerate}
  \item Prove that\footnote{\emph{Hint}: Use the formula for the sum
      of a finite geometric series and use the formula
      \(\sin x = \frac{1}{2i} (e^{ix} - e^{-ix})\).}
    \begin{equation}
      \label{eq-Dirichlet}
      D_N(x)
      = \frac{\sin\Bracket{\Paren{N + \frac 1 2} x}}{\sin \frac x 2}.
    \end{equation}

    \begin{solution}
      \begin{proof}

      \end{proof}
    \end{solution}

  \item Prove that
    \(\displaystyle \frac{1}{2\pi} \int_{-\pi}^{\pi} D_N(x) \dif x =
    1\) for all \(N \in \N\).

    \begin{solution}
      \begin{proof}

      \end{proof}
    \end{solution}

  \item The \emph{Lebesgue constants} \(L_N\) are defined by
    \begin{equation*}
      L_N
      = \frac{1}{2\pi} \int_{-\pi}^{\pi} \Abs{D_N(x)} \dif x
      = \frac{1}{2\pi} \int_{-\pi}^{\pi}
      \Abs{\frac{\sin\Bracket{\Paren{N + \frac 1 2}x}}
        {\sin \frac x 2}} \dif x.
    \end{equation*}
    Prove that \(L_N \geq c \log N\) for some constant
    \(c > 0\).\footnote{\emph{Hint}: Use symmetry and the substitution
      \(u = \frac x 2\), then use the inequality \(\sin u \leq u\) for
      \(u \in \Bracket{0,\frac \pi 2}\).  You will also need to
      ``break up'' the resulting integral and then use the Integral
      Test from Calculus II to estimate the partial sums of the
      harmonic series.}  This implies that the Dirichlet kernel is not
    a summability kernel.\footnote{The fact that \(L_N \to \infty\)
      can be used to prove that there exist continuous functions whose
      Fourier series diverge at certain points.}

    \begin{solution}
      \begin{proof}

      \end{proof}
    \end{solution}
  \end{enumerate}
\end{Exercise}
\end{document}
