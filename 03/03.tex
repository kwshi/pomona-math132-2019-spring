\documentclass{homework}

\homework{3}
\date{Thursday 2/14}
\author{}
\coauthor{}

\begin{document}
\begin{Exercise}
	This exercise continues our exploration of normed vector spaces.
  This material will be important later on.
	\begin{enumerate}
  \item Prove that if $\W$ is a subspace (in the linear algebra sense)
    of a normed vector space $\V$, then so is its closure
    $\W^- $.\footnote{Hint: Since $\W^- $ is a nonempty (it contains
      the zero vector), it suffices to prove that $\W^- $ is closed
      under addition and scalar multiplication.  }

    \begin{solution}

    \end{solution}

  \item If $\W$ is a proper, closed (in the topological sense)
    subspace (in the linear algebra sense) of a normed vector space
    $\V$, show that $\W$ is nowhere dense in $\V$.\footnote{Hint:
      Suppose toward a contradiction that $\W$ is not nowhere dense in
      $\V$.  In other words, suppose that $\W^- = \W$ has nonempty
      interior (i.e., it contains a nonempty open set).  Use this to
      deduce that $\W = \V$, a contradiction.} \footnote{Remark: The
      hypothesis that $\W$ is topologically \emph{closed} is extremely
      important.  For instance, the Weierstrass Approximation Theorem
      asserts that the subspace $\mathcal{P}$ of all polynomials is
      dense in $C[0,1]$ with respect to the $d_{\infty}$ metric (i.e.,
      the ``sup metric'').  This illustrates a fundamental distinction
      between finite-dimensional vector space theory (i.e., linear
      algebra) and the infinite-dimensional theory (i.e., functional
      analysis).  For instance, it is clearly impossible for a proper
      subspace of $\R^3$ to be dense in $\R^3$!}

    \begin{solution}

    \end{solution}

  \item Let $\V$ be an infinite-dimensional normed vector space and
    suppose that $\V = \bigcup_{n=1}^{\infty} \V_n$, in which each
    $\V_n$ is a finite-dimensional subspace of $\V$.  Prove that $\V$
    is not complete.

    \begin{solution}

    \end{solution}

  \item Is it possible to define a norm on the vector space
    $\mathcal{P}$ of all polynomials such that $\mathcal{P}$ is a
    complete metric space with respect to the induced metric
    $d(p,q) = \norm{p-q}$?

    \begin{solution}

    \end{solution}
	\end{enumerate}
\end{Exercise}

\begin{Exercise}
	Use the Weierstrass Approximation Theorem to help you prove that
  $C[0,1]$ is separable (i.e., has a countable dense subset).

  \begin{solution}

  \end{solution}
\end{Exercise}

\begin{Exercise}
	Let $(X,d)$ be a compact metric space and let $f_n: X \to \R$ be a
  sequence of continuous functions.  Suppose that
  $f_n(x) \leq f_{n+1}(x)$ holds for all $x \in X$ and $n \in \N$.
  Prove that if $f_n$ converges pointwise to a continuous function
  $f: X \to \R$, then $f_n$ actually converges uniformly to
  $f$.\footnote{Hint: Let $g_n = f - f_n$ and note that
    $\lim_{n\to\infty} g(x) = 0$ for all $x \in X$.  Let
    $\epsilon > 0$ be given and let
    $S_n = g_n^{-1}(-\infty,\epsilon)$.  Now observe that
    $X = \bigcup_{n=0}^{\infty} S_n$.}%
	This result is known as \emph{Dini's Theorem}.

  \begin{solution}

  \end{solution}
\end{Exercise}

\begin{Exercise}
	The \emph{Baire-Osgood Theorem} asserts that if $f_n:\R\to\R$ is a
  sequence of continuous functions that converges pointwise to a
  function $f:\R\to\R$, then $DC(f)$ is a set of first category in
  $\R$ (i.e., $DC(f)$ is the countable union of nowhere dense sets).
  In particular, this implies that $f$ is continuous on a dense subset
  of $\R$.  From lecture, we know that $DC(f)$ can be represented as
  the countable union of closed sets:
	\begin{equation*}
		DC(f) = \bigcup_{n=1}^{\infty}
    \{x\in \R: \omega_f(x) \geq \tfrac{1}{n} \}
	\end{equation*}
	Therefore to prove the Baire-Osgood Theorem, it suffices to prove
  that for every $\epsilon > 0$, the closed set
	\begin{equation*}
		S = \{x \in \R : \omega_f(x) \geq \epsilon\}
	\end{equation*}
	is nowhere dense in $\R$.
	\begin{enumerate}
  \item  Consider the sets
    \begin{equation*}
      E_n = \bigcap_{i,j \geq n}\{ x \in \R :
      |f_i(x) - f_j(x)| \leq \tfrac{\epsilon}{5} \}.
    \end{equation*}
    Prove that each $E_n$ is closed and that
    $\bigcup_{n=1}^{\infty} E_n =\R$.\footnote{Hint: Consider the
      functions $g_{ij}(x) = f_i(x) - f_j(x)$ and use the inverse
      image characterization of continuity.}

    \begin{solution}

    \end{solution}

  \item Suppose that $I = [a,b]$ (where $a<b$) is a closed interval.
    Prove that there exists $n \in \N$ such that $E_n \cap I$
    contains a nonempty open interval $J$.\footnote{Hint: Use the
      fact that $\bigcup_{n=1}^{\infty} E_n =\R$.  You will need the
      Baire Category Theorem for this problem.}

    \begin{solution}

    \end{solution}

  \item Prove that\footnote{Hint: This requires (ii) and several
      applications of the Triangle Inequality.  Start with the
      observation that since $J \subseteq E_n$ by (ii), we have
      $|f_i(x) - f_j(x)| \leq \tfrac{\epsilon}{5}$ for all $x \in J$
      and all $i,j \geq n$.  Eventually, you want to show that for
      each $x_0 \in J$, there exists an interval $I_0 \ni x_0$ such
      that
      $\omega_f(x_0) \leq \omega(f, I_0) \leq \tfrac{4\epsilon}{5} <
      \epsilon$. }
    \begin{equation*}
      J \subseteq S^c = \{x \in \R : \omega_f(x) < \epsilon\}.
    \end{equation*}

    \begin{solution}

    \end{solution}

  \item Conclude the proof of the Baire-Osgood Theorem by showing
    that $S$ is nowhere dense in $\R$.

    \begin{solution}

    \end{solution}

  \item Use the Baire-Osgood Theorem to prove that if $f:\R\to\R$ is
    differentiable, then $DC(f')$ is a set of first
    category.\footnote{Hint: Express $f'(x)$ as a limit of
      continuous functions.}

    \begin{solution}

    \end{solution}

  \item Use the Baire-Osgood Theorem to prove that $\chi_{\Q}$ is not
    the pointwise limit of a sequence of continuous functions.

    \begin{solution}

    \end{solution}
  \end{enumerate}
	It is interesting to note that $\chi_{\Q} = \lim_{k \to \infty} f_k$
  where
	$$f_k(x) = \lim_{n\to\infty} ( \cos (k!x \pi ) )^{2n}$$ for $k \in \N$.  Thus, $\chi_{\Q}$ is the pointwise limit
	of a sequence of functions, each of which is the pointwise limit of
  continuous functions.  One says that $\chi_{\Q}$ is not of
  \emph{Baire class} 1, but it is of Baire class 2.\footnote{i.e., it
    takes two consecutive pointwise limiting procedures to obtain
    $\chi_{\Q}$ from continuous functions.}  Continuing with this
  terminology in the obvious way, Lebesgue (1905) proved that distinct
  Baire classes exist for each $n\in \N$.  In fact, more is true --
  however this would require a digression into transfinite arithmetic
  and the theory of ordinal numbers.
\end{Exercise}
\end{document}