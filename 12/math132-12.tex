\documentclass{../homework}

\homework{12}
\date{Thursday 4/25}

\author{}
\coauthor{}

\begin{document}
\begin{Exercise}
  Define \(f\colon\T\to\C\) by
  \[
    f(t) =
    \begin{cases}
      0 & \text{if \(-\pi < x < 0\),} \\
      \pi & \text{if \(0 < x < \pi\).}
    \end{cases}
  \]
  For the moment, let \(f\) be undefined at \(x = 0\) and \(\pi\).
  \begin{enumerate}
  \item Compute the real Fourier series of \(f\)

    \begin{solution}

    \end{solution}

  \item Use (i) to derive \emph{Leibniz' Formula} (1673):
    \(\displaystyle 1 - \frac{1}{3} + \frac{1}{5} - \frac{1}{7} +
    \cdots = \frac{\pi}{4}\).

    \begin{solution}
      \begin{proof}

      \end{proof}
    \end{solution}

  \item Use (i) to show that
    \[
      \sum_{n=1}^{\infty} \frac{1}{(2n-1)^2}
      = 1 + \frac{1}{3^2} + \frac{1}{5^2} + \frac{1}{7^2} + \cdots
      = \frac{\pi^2}{8}.
    \]

    \begin{solution}
      \begin{proof}

      \end{proof}
    \end{solution}

  \item Use the preceding problem to show that
    \[
      1 + \frac{1}{4} + \frac{1}{9} + \frac{1}{16} + \cdots
      = \frac{\pi^2}{6}.
    \]
    Euler discovered this formula in 1736 using a different method.

    \begin{solution}
      \begin{proof}

      \end{proof}
    \end{solution}
  \end{enumerate}
\end{Exercise}

\begin{Exercise}
  A sequence \(\xi_1,\xi_2,\ldots\) in \([0,1)\) is
  \emph{equidistributed} if for each \((a,b) \subset [0,1)\),
  \[
    \lim_{N \to \infty}
    \frac{\#\{1 \leq n \leq N : \xi_n \in (a,b)\}}{N} = b-a.
  \]
  That is, for large \(N\) the proportion of the \(\xi_n\) in the
  interval \((a,b)\) is comparable to \(b-a\).
  \begin{enumerate}
  \item Prove \emph{Weyl's Equidistribution Theorem}: If \(\xi\) is
    irrational, then the sequence
    \(\inner{\xi}, \inner{2\xi}, \inner{3 \xi}, \ldots\) of fractional
    parts\footnote{The \emph{fractional part} \(\inner{x}\) of
      \(x \in \R\) is defined by \(x - [x]\) where \([x]\) denotes the
      greatest integer \(\leq x\).  Therefore \(0 \leq \inner{x} < 1\)
      for all \(x \in \R\).}  of \(n\xi\) is equidistributed in
    \([0,1)\).\footnote{\emph{Hint}: Observe that Weyl's Theorem is
      equivalent to the statement
      \( \lim_{N\to\infty} \frac{1}{N} \sum_{n=1}^N \chi_{(a,b)}(n
      \xi) = \int_0^1 \chi_{(a,b)}(x)\,dx = b-a\).  for all
      \((a,b) \subset [0,1)\).  Approximate \(\chi_{(a,b)}\) with
      continuous piecewise linear periodic functions
      \(f_{\epsilon}^+\) and \(f_{\epsilon}^-\) of period \(1\) which
      approximate \(\chi_{(a,b)}(x)\) on \([0,1)\) from above and
      below and which agree with \(\chi_{(a,b)}\) on \((a,b)\) and
      \((a+ \epsilon, b- \epsilon)\), respectively.}

    \begin{solution}
      \begin{proof}

      \end{proof}
    \end{solution}

  \item Prove \emph{Kronecker's Theorem}: If \(\xi\) is irrational,
    then the sequence
    \(\inner{\xi}, \inner{2\xi}, \inner{3 \xi}, \ldots\) of fractional
    parts of \(n\xi\) is dense in \([0,1)\).

    \begin{solution}
      \begin{proof}

      \end{proof}
    \end{solution}

  \end{enumerate}
\end{Exercise}
\end{document}
