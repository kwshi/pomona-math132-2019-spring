\documentclass{homework}

\homework{1}
\date{Thursday 1/31}
\author{}
\coauthor{}

\begin{document}
\begin{Exercise}
	This exercise guides you through the construction of a continuous
	function on $\R$ that is nowhere differentiable.
	\begin{enumerate}
  \item Prove that if $f:\R\to\R$ is differentiable at $x$, then
    \begin{equation*}
      \lim_{n\to\infty} \frac{ f(v_n) - f(u_n) }{v_n - u_n} = f'(x)
    \end{equation*}
    whenever $u_n,v_n$ are sequences such that the following
    hold\footnote{Hint: You will need to use the definition of the
      derivative as a limit of difference quotients.  Without loss of
      generality, you can also assume that $f'(x) = 0$ (why?).}
    \smallskip
    \begin{enumerate}
    \item $u_n \leq x < v_n$ for all $n \in \N$,
    \item $u_n <v_n$ for all $n \in \N$,
    \item $\lim_{n\to\infty} (v_n - u_n) = 0$.
    \end{enumerate}

    \begin{solution}
    \end{solution}

  \item Given $x \in \R$, let $g(x)$ denote the distance from $x$ to
    the nearest integer.  Prove that the function $f:\R\to\R$ defined
    by
    \begin{equation*}
      f(x) = \sum_{n=0}^{\infty} \frac{g(2^nx)}{2^n}
    \end{equation*}
    is well-defined, continuous, and bounded.\footnote{Hint: Use the
      \emph{Weierstrass $M$-Test} to show that the series converges
      uniformly on $\R$, then recall that the uniform limit of
      continuous functions is continuous.}

    \begin{solution}
    \end{solution}

  \item Prove that the function $f$ from (ii) is not differentiable at
    any $x \in \R$.\footnote{Hint: First observe that if $x$ is a
      dyadic rational number (i.e., $x = \tfrac{m}{2^n}$ for some
      $m \in \Z$ and $n \in \N$), then $2^k x \in \Z$ whenever
      $k \geq n$ and thus $g(2^kx) = 0$ for all $k \geq n$.  Now fix
      $x \in \R$ and for each $n\in \N$ let $u_n = \frac{m_n}{2^n}$
      and $v_n = \frac{m_n+1}{2^n}$ be dyadic rational numbers which
      satisfy $u_n \leq x < v_n$ and $v_n - u_n = \frac{1}{2^n}$.  Now
      consider the expression $\frac{ f(v_n) - f(u_n) }{v_n -u_n}$ and
      use (i).}

    \begin{solution}
    \end{solution}
	\end{enumerate}
\end{Exercise}

\begin{Exercise}
	A function $f:[a,b] \to \R$ is called \emph{piecewise linear} if the
  graph of $f$ consists of finitely many line segments.  Let
  $CPL[a,b]$ denote the set of all continuous piecewise linear
  functions on $[a,b]$.  In particular, note that $CPL[a,b]$ is a
  subspace of the vector space $C[a,b]$.\footnote{Recall from Math 131
    that $C[a,b]$ denotes the set of all continuous functions
    $f:[a,b]\to\R$.  Since $C[a,b]$ is closed under addition and
    scalar multiplication, it is a vector space.  In fact, $C[a,b]$ is
    a \emph{Banach space} with respect to the \emph{norm}
    $\norm{f}_{\infty} = \sup_{x\in[a,b]} |f(x)|$.  What does this
    mean?  First of all, the expression $\norm{f}_{\infty}$ is finite
    for each $f$ in $C[a,b]$.  Why?  Because $[a,b]$ is compact and
    $f:[a,b]\to\R$ is continuous it follows that $f$ is bounded
    (\emph{Extreme Value Theorem}).  The norm $\norm{f}_{\infty}$
    defines a \emph{metric} $d_{\infty}(f,g) = \norm{f-g}_{\infty}$ on
    $C[a,b]$.  Observe that convergence with respect to the metric
    $d_{\infty}$ is equivalent to \emph{uniform convergence} on
    $[a,b]$.  It is proven in many Math 131 courses that $C[a,b]$ is a
    \emph{complete} metric space with respect to the metric
    $d_{\infty}$.  However, $C[a,b]$ is even better -- it is a
    \emph{Banach space} (a vector space, endowed with a norm, which is
    complete with respect to the metric induced by the norm).}
	\begin{enumerate}
  \item If $(x,y)$ is a ``corner'' or ``endpoint'' of the graph of
    $f \in CPL[a,b]$, then $x$ is said to be a \emph{node} of $f$.
    Let $a = x_1 < x_2 < \cdots < x_n = b$ be distinct points in
    $[a,b]$ and let $S$ denote the set of all functions in $CPL[a,b]$
    with nodes precisely at the $x_i$.  It turns out that $S$ is an
    $n$-dimensional subspace of $C[a,b]$ and that the constant
    function $\phi_0(x) = 1$ along with the $n-1$ functions
    \begin{equation*}
      \phi_i(x) = | x - x_i| + (x - x_i)
    \end{equation*}
    for $i = 1,2,\ldots, n-1$ form a basis for $S$.  Briefly
    explain\footnote{i.e., this does not require rigorous proof.  Just
      demonstrate that you understand the principle.}  why this is
    true.\footnote{Hint: Consider the graphs of the $\phi_i$.  How
      does their shape help determine the coefficients in the
      expression $f = c_0 \phi_0 + \cdots + c_{n-1} \phi_{n-1}$ for
      $f \in CPL[a,b]$?}

    \begin{solution}
    \end{solution}

  \item Prove that any function in $CPL[a,b]$ can be written in the
    form\footnote{This representation will be important later on when
      we discuss the Weierstrass Approximation Theorem.}
    \begin{equation*}
      \left( \sum_{i=1}^m \alpha_i | x - x_i| \right) + \beta x +
      \gamma
    \end{equation*}
    for constants $\alpha_i,\beta,\gamma$ and some natural number $m$.

    \begin{solution}
    \end{solution}

  \item Prove that $CPL[a,b]$ is dense in $C[a,b]$.\footnote{Hint:
      Since $[a,b]$ is compact, it follows that $f$ is uniformly
      continuous and hence there exists $\delta > 0$ such that
      $|x - y| < \delta$ implies $|f(x) - f(y)| < \tfrac{\epsilon}{2}$
      for all $x,y \in [a,b]$.  Choose a natural number
      $n > \frac{b-a}{\delta}$ and partition $[a,b]$ into $n$ equal
      subintervals $I_j = [x_{i-1},x_i]$ (i.e.,
      $x_i = \frac{(b-a)i}{\delta}$ for $i=0,1,2,\ldots,n$).  Use this
      to construct a piecewise linear function $\phi$ such that
      $\norm{f - \phi}_{\infty} < \epsilon$}

    \begin{solution}
    \end{solution}

  \item A metric space $(M,d)$ is \emph{separable} if there exists a
    countable subset of $M$ which is dense in $M$.  Briefly explain
    why $C[a,b]$ is separable.

    \begin{solution}
    \end{solution}

	\end{enumerate}
\end{Exercise}
\end{document}