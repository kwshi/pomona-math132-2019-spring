\documentclass{../homework}

\homework{11}
\date{Thursday 4/18}

\author{}
\coauthor{}

\begin{document}
\begin{Exercise}
  Let \(\xi\) be an irrational number.  Prove that if \(f:\R \to \C\)
  is continuous and periodic with period \(1\), then\footnote{This
    result is an important stepping stone to Weyl's Equidistribution
    Theorem from number theory.  One way of thinking of
    \eqref{eq-WeylIntegral} is that irrational rotations ``mix'' the
    circle \(\T\) so thoroughly that ``time averages'' equal ``space
    averages'' for continuous \(f\).}  \footnote{\emph{Hint}: First
    prove the result for complex exponentials
    \(f(x) = e^{2\pi i k x}\) for \(k \in \Z\).  By linearity,
    conclude that \eqref{eq-WeylIntegral} holds for all trigonometric
    polynomials (with \(e^{2\pi i kx}\) in place of \(e^{ikx}\) for
    scaling reasons).}
  \begin{equation}
    \label{eq-WeylIntegral}
    \lim_{N \to \infty} \frac{1}{N}
    \sum_{n=1}^N f(n \xi) = \int_0^1 f(x) \dif x.
  \end{equation}

  \begin{solution}
    \begin{proof}

    \end{proof}
  \end{solution}
\end{Exercise}

\begin{Exercise}
  This exercise concerns some results concerning the relationship
  between the smoothness of a function and the rate of decay of its
  Fourier coefficients.
  \begin{enumerate}
  \item Prove that if \(f \in C^1(\T)\), then
    \(\widehat{f'}(n) = in \hat f(n)\) for all
    \(n \in \Z\).\footnote{\emph{Hint}: Integrate the definition of
      \(\hat{f}(n)\) by parts.}

    \begin{solution}
      \begin{proof}

      \end{proof}
    \end{solution}

  \item Prove that if \(f \in C^r(\T)\), then
    \(\hat f(n) = o\Paren{\frac{1}{\abs n^r}}\) as
    \(n \to \pm \infty\).\footnote{\(F(n) = o(G(n))\) means that
      \(\lim_{n\to\infty} \frac{F(n)}{G(n)} = 0\).  That is, \(F\)
      goes to zero faster than \(G\) does.}

    \begin{solution}
      \begin{proof}

      \end{proof}
    \end{solution}

  \item Prove that if \(f \in C^2(\T)\), then the Fourier series for
    \(f\) converges absolutely and uniformly to \(f\).

    \begin{solution}
      \begin{proof}

      \end{proof}
    \end{solution}
  \end{enumerate}
\end{Exercise}


\begin{Exercise}
  Let \(f(x) = x^2\) for \(\abs x \leq \pi\) and extend \(f\) by
  \(2\pi\)-periodicity to all of \(\R\).
  \begin{enumerate}
  \item Compute the real Fourier series of this function.

    \begin{solution}

    \end{solution}

  \item Prove that the Fourier series for \(f\) converges uniformly to
    \(f\).\footnote{Note that \(f \notin C^2(\T)\) since there are
      ``cusps'' at \(x = \pm \pi\).}

    \begin{solution}
      \begin{proof}

      \end{proof}
    \end{solution}

  \item Use (i) and (ii) to prove that\footnote{The sum of this famous
      series was first computed by Euler.}
    \(\displaystyle \sum_{n=1}^{\infty} \frac{1}{n^2} =
    \frac{\pi^2}{6}\).

    \begin{solution}
      \begin{proof}

      \end{proof}
    \end{solution}

  \item Use Parseval's Formula and (i) to prove that\footnote{Euler
      computed \(\sum_{n=1}^{\infty} \frac{1}{n^{2k}}\) for
      \(k = 1,2,3,\ldots\).  An explicit formula for
      \(\sum_{n=1}^{\infty} \frac{1}{n^3}\) is still unknown!}
    \(\displaystyle \sum_{n=1}^{\infty} \frac{1}{n^4} =
    \frac{\pi^4}{90}\).

    \begin{solution}
      \begin{proof}

      \end{proof}
    \end{solution}
  \end{enumerate}
\end{Exercise}
\end{document}
