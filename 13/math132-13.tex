\documentclass{../homework}

\homework{13}
\date{Thursday 5/2}

\author{}
\coauthor{}

\begin{document}
\begin{Exercise}
  Let \(f(x) = \cos zx\) for \(x \in [-\pi,\pi]\) and suppose that
  \(z \notin \Z\).
  \begin{enumerate}
  \item Compute the real Fourier series of \(f\).

    \begin{solution}

    \end{solution}

  \item Prove that the Fourier series for \(f\) converges uniformly to
    \(f\).

    \begin{solution}
      \begin{proof}

      \end{proof}
    \end{solution}

  \item Use (i) and (ii) to prove
    \begin{equation}\label{eq-Cot}
      \pi \cot \pi z
      = \frac{1}{z} + 2z \sum_{n=1}^{\infty} \frac{1}{z^2 - n^2}
    \end{equation}
    for \(z \notin \Z\).  The formula \eqref{eq-Cot} is known as the
    \emph{partial fractions} expansion of
    \(\pi \cot \pi z\).\footnote{Infinite partial fraction expansions
      are something discussed in graduate-level complex analysis.}

    \begin{solution}
      \begin{proof}

      \end{proof}
    \end{solution}

  \item Use (iii) to prove the formula
    \begin{equation}\label{eq-Sine}
      \frac{\sin \pi z}{\pi z}
      = \prod_{n=1}^{\infty} \Paren{1 - \frac{z^2}{n^2}}
    \end{equation}
    for \(z \notin \Z\).\footnote{\emph{Hint}: For each \(n \in \N\),
      the series \eqref{eq-Cot} converges uniformly
      \(r \leq \abs z < R\) if \(r,R\) are chosen correctly.  On such
      a region (a closed annulus in the complex plane) the series can
      then be integrated term-by-term.} \footnote{Infinite product
      factorizations of functions are explored further in
      graduate-level complex analysis.}

    \begin{solution}
      \begin{proof}

      \end{proof}
    \end{solution}

  \item Prove \emph{Wallis' Formula}\footnote{Hint: Use (iv).}
    \begin{equation*}
      \frac{\pi}{2} = \frac{2}{1} \cdot \frac{2}{3}
      \cdot \frac{4}{3} \cdot \frac{4}{5}
      \cdot \frac{6}{5} \cdot \frac{6}{7} \cdot \cdots.
    \end{equation*}

    \begin{solution}
      \begin{proof}

      \end{proof}
    \end{solution}
  \end{enumerate}
\end{Exercise}

\begin{Exercise}
  For \(0 \leq r < 1\), the \emph{Poisson kernel} is the function
  \(P_r(t) = \sum_{n \in \Z} r^{\abs n } e^{int}\).  In other words,
  \(P_r(t) = 1 + 2 \sum_{n=1}^{\infty} r^n \cos n t\) for
  \(n \in \Z\).
  \begin{enumerate}
  \item Prove that\footnote{Hint: Use the formula for the sum of a
      finite geometric series and use Euler's Formula to show that
      \(\cos x = \frac{1}{2} (e^{ix} + e^{-ix})\).}
    \begin{equation*}
      P_r(t) = \frac{1 - r^2}{1 - 2r \cos t + r^2}.
    \end{equation*}

    \begin{solution}
      \begin{proof}

      \end{proof}
    \end{solution}

  \item Graph \(P_r(t)\) for \(t \in [-\pi,\pi]\) and several values
    of \(r\).  Describe the behavior of \(P_r(t)\) as
    \(r \to 1^-\).\footnote{You may wish to use a computer for this
      problem.}

    \begin{solution}

    \end{solution}

  \item Prove that \(P_r(t)\) is a (continuous) summability
    kernel.\footnote{By a ``continuous'' summability kernel, we mean
      that \(P_r(t)\) satisfies the definition of a summability kernel
      except that this family is indexed by the continuous parameter
      \(r \in [0,1)\), rather than by a discrete parameter
      \(n \in \N\).}

    \begin{solution}
      \begin{proof}

      \end{proof}
    \end{solution}

  \item Prove that if \(f\colon \T\to\C\) is Riemann integrable then
    \begin{equation*}
      \lim_{r\to 1^-} \frac{1}{2\pi} \int_{-\pi}^{\pi}
      \frac{1 - r^2}{1 - 2r \cos (\theta - t) + r^2} f(e^{it}) \dif t
      = f(e^{i \theta})
    \end{equation*}
    for all \(\theta\) at which \(f\) is continuous.  Prove that if
    \(f \in C(\T)\), then the convergence is uniform.

    \begin{solution}
      \begin{proof}

      \end{proof}
    \end{solution}
  \end{enumerate}

  Using basic complex analysis or tedious multivariable calculus, it
  is possible to prove that \emph{Poisson's Formula}
  \begin{equation}\label{eq-Extension}
    f(re^{i\theta}) = \frac{1}{2\pi} \int_{-\pi}^{\pi}
    \frac{1 - r^2}{1 - 2r \cos(\theta - t) + r^2} f(e^{it}) \dif tt,
    \quad r \in [0,1)
  \end{equation}
  yields an extension of a function \(f\colon \T\to\R\) (originally
  defined on the unit circle) to a function defined on the unit disk
  and satisfying the boundary value problem
  \begin{equation*}
    \pd[2] u x + \pd[2] u y
    \qquad u(e^{i\theta}) = f(e^{i\theta}).
  \end{equation*}
  In physical terms, \eqref{eq-Extension} solves the
  \emph{steady-state heat equation} \(\pd[2] u x + \pd[2] u y = 0\) on
  \(\mathbb{D}\) subject to the temperature function \(f\) applied to
  the boundary \(\T\)
\end{Exercise}
\end{document}
