\documentclass{../homework}

\homework 6
\date{Thursday 3/7}

\author{}
\coauthor{}

\begin{document}
\begin{Exercise}
	Let $p_n$ be a sequence of polynomials of degree $\leq d$ so that
  $\lim_{n\to\infty} p_n(x) = 0$ for all $x\in [0,1]$.  Prove that
  $p_n \rightrightarrows 0$ on $[0,1]$.\footnote{Hint: Construct a
    convenient basis for the vector space of polynomials of degree
    $\leq d$.}
\end{Exercise}
\begin{solution}
  \begin{proof}

  \end{proof}
\end{solution}

\begin{Exercise}
	Let $f_n \in C[0,1]$ and suppose that
	\begin{equation*}
		\int_0^1 |f_n(x)|^2\,dx \leq \frac{2}{3}
	\end{equation*}
	for all $n \in \N$.  Let $g_n:[0,1] \to \R$ be defined by
	\begin{equation*}
		g_n(x) = \int_0^1f_n(y)  \sqrt{x+y}\, dy.
	\end{equation*}
	Prove that some subsequence of the $g_n$ converges uniformly on
  $[0,1]$.\footnote{Use the Arzel\`a--Ascoli Theorem.  The
    Cauchy--Schwarz inequality for integrals may be useful.}
\end{Exercise}
\begin{solution}
  \begin{proof}

  \end{proof}
\end{solution}

\begin{Exercise}
	Let $f_n \in C^2(\R)$ satisfy $f_n(0) = f_n'(0) = 0$ for all
  $n \in \N$.  Suppose that $|f_n''(x)| \leq 1$ for all $n\in \N$ and
  $x \in \R$.  Prove that there is a subsequence of $f_n$ which
  converges pointwise on $\R$.\footnote{Hint: Use Arzel\`a-Ascoli to
    prove that for each $M \in \N$, some subsequence of $f_n$
    converges uniformly on $[-M,M]$.  Use a ``diagonal argument'' to
    construct a subsequence of the $f_n$ that converges on $\R$.  The
    application of Arzel\`a-Ascoli to $[-M,M]$ is difficult --
    Taylor's Formula (with remainder term) and the Mean Value Theorem
    will come in handy!}
\end{Exercise}
\begin{solution}
  \begin{proof}

  \end{proof}
\end{solution}

\begin{Exercise}
	Let $f:\R\to \R$ be defined by
	\begin{equation*}
		f(x) =
		\begin{cases}
			0 & x \notin \Q, \\
			1 & x =0, \\
			\frac{1}{q} & \text{ if $x = \frac{p}{q}$},
		\end{cases}
	\end{equation*}
	in which $x = p/q$ refers to a rational number expressed in lowest
  terms and $q > 0$.
	\begin{enumerate}
  \item Use the definition of the Riemann integral and the Refinement
    Lemma to prove that $f$ is Riemann integrable on $[0,1]$.

    \begin{solution}
      \begin{proof}

      \end{proof}
    \end{solution}

	\item Prove the same result using the Riemann--Lebesgue Theorem.

    \begin{solution}
      \begin{proof}

      \end{proof}
    \end{solution}
  \end{enumerate}
\end{Exercise}
\end{document}
